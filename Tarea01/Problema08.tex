\textbf{8.} Suponga que se tiene una lista de aristas doblemente ligada (DCEL) de una subdivisión
plana. Diseña un algoritmo que encuentre todas las caras que contienen un vértice en la cara exterior.
\newline

\textbf{Solución.} A continuación se exhibe el algoritmo requerido:
\begin{enumerate}
\item Ordenamos la lista de aristas (DCEL), esto nos toma $\mathcal{O}(n \log n)$.
\item Encontremos el punto más ``chico'' en nuestro polígono. Esto lo realizamos en $\mathcal{O}(1)$,
  pues hemos ordenado nuestras aristas.
\item Recorramos siempre hacia la izquierda excepto cuando:
  \begin{enumerate}
  \item El vértice al que llegamos no pertenece a una cara externa. En este caso nos regresamos
    y ahora tomamos el camino más a la derecha.
  \item El vértice de llegada es el vértice de partida, entonces hemos llegado y terminamos.
  \end{enumerate}
  Durante el recorrido vamos acumulando los vértices que sean externos en una lista, como el recorrido
  esta dado por las aristas DCEL esto nos toma a lo más tiempo $\mathcal{O}(n)$. Pues cuando lleguemos a un vértice
  interno de inmediato nos regresamos.
  
  ¿Cómo sabemos que el vértice pertenece o no a la cara externa? La arista nos lo indica.
\end{enumerate}

\textit{Análisis de complejidad}. La complejidad de este algoritmo esta contenida en
\[\mathcal{O}(n \log n) + \mathcal{O}(n) + \mathcal{O}(1) = \mathcal{O}(n \log n).\]

