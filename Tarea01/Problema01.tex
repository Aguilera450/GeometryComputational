\textbf{1.} Sea $S$ un conjunto de puntos en el plano en posición general. Demuestra que el
cierre convexo de $S$ es el polígono convexo con perímetro y área más pequeños que contienen a $S$.
\newline

\textbf{\textit{Dem.}} Para este problema, dividamos la prueba en dos posibles casos:
\begin{itemize}
\item Con $C(S)$\footnote{Diremos que $C$ es la función que nos devuelve como imagen el cierre convexo del conjunto de puntos
que se pase como argumento.} el cierre convexo. $C(S)$ es el polígono convexo con perímetro más pequeño que contiene a $S$.
  \newline
  Prueba por \textit{reducción al absurdo}. Supongamos que $C(S)$ no es el polígono de perímetro mínimo que envuelve
  todos los puntos en $S$. La suposición anterior implica que existe $C(S)' \not= C(S)$ tal que $P(C(S)') < P(C(S))$\footnote{Diremos
  que $P$ es la función que se mapea al perímitro del polígono que se pasa como argumento.} con $C(S)'$ un segundo cierre convexo. Entonces,
  ¿Cómo modificar $C(S)$ para convertirlo en $C(S)'$? Esto puede ocurrir en 3 casos, estos son:
  \newline
  
  \hspace*{0.5cm} \textit{Caso 1.} Existe un punto interno al polígono que lo convierte en uno de menor perímetro,
  en este caso perdemos convexidad en la nueva figura y por tanto debemos descartar este caso (pues esta condición es parte
  del antecedente de nuestra implicación).
  %%%%%%%%%%%%%%%%%%%%%%%%%%%%%%%%%%%%%%%%%%%%%%%%%%%%%%%%%%%%%%%%%%%%%%%%%% FIGURE 1
  \begin{figure}[ht!]
    \centering
    \begin{tikzpicture}
      \node(0) [blueV, label=180:$v_{i-4}$] at (-4, 3){};
      \node(1) [blueV, label=180:$v_{i-3}$] at (-3, 4){};
      \node(2) [blueV, label=150:$v_{i-2}$] at (-2, 4.5){};
      \node(3) [blueV, label=150:$v_{i-1}$] at (-1, 4.8){};
      \node(4) [blueV, label=90:$v_i$] at (0, 5){};
      \node(5) [blueV, label=0:$v_{i+4}$] at (4, 3){};
      \node(6) [blueV, label=0:$v_{i+3}$] at (3, 4){};
      \node(7) [blueV, label=30:$v_{i+2}$] at (2, 4.5){};
      \node(8) [blueV, label=30:$v_{i+1}$] at (1, 4.8){};
      \node(ext) [redV, label=270:$v_{j}$] at (0.5, 4){};
      \draw[edgeC] (0) to (5);
      \draw[edge] (0) to (1);
      \draw[edge] (1) to (2);
      \draw[edge] (2) to (3);
      \draw[edge] (3) to (4);
      \draw[edge] (4) to (8);
      \draw[edge] (5) to (6);
      \draw[edge] (6) to (7);
      \draw[edge] (7) to (8);
      \draw[edgeDotted] (ext) to (4);
      \draw[edgeDotted] (ext) to (8);
    \end{tikzpicture}
  \end{figure}
  %%%%%%%%%%%%%%%%%%%%%%%%%%%%%%%%%%%%%%%%%%%%%%%%%%%%%%%%%%%%%%%%%%%%%%%%%%
  
  \hspace*{0.5cm} \textit{Caso 2.} Existe un punto, extra en $C(S)'$ y que no esta en $C(S)$, sobre un segmento
  de $C(S)$, esto no reduce el perímetro y por tanto podemos descartar este caso.
  %%%%%%%%%%%%%%%%%%%%%%%%%%%%%%%%%%%%%%%%%%%%%%%%%%%%%%%%%%%%%%%%%%%%%%%%%% FIGURE 2
  \begin{figure}[ht!]
    \centering
    \begin{tikzpicture}
      \node(0) [blueV, label=180:$v_{i-4}$] at (-4, 3){};
      \node(1) [blueV, label=180:$v_{i-3}$] at (-3, 4){};
      \node(2) [blueV, label=150:$v_{i-2}$] at (-2, 4.5){};
      \node(3) [blueV, label=150:$v_{i-1}$] at (-1, 4.8){};
      \node(4) [blueV, label=90:$v_i$] at (0, 5){};
      \node(5) [blueV, label=0:$v_{i+4}$] at (4, 3){};
      \node(6) [blueV, label=0:$v_{i+3}$] at (3, 4){};
      \node(7) [blueV, label=30:$v_{i+2}$] at (2, 4.5){};
      \node(8) [blueV, label=30:$v_{i+1}$] at (1, 4.8){};
      \node(ext) [redV, label=270:$v_{j}$] at (0.5, 4.95){};
      \draw[edgeC] (0) to (5);
      \draw[edge] (0) to (1);
      \draw[edge] (1) to (2);
      \draw[edge] (2) to (3);
      \draw[edge] (3) to (4);
      %\draw[edge] (4) to (8);
      \draw[edge] (4) to (ext);
      \draw[edge] (ext) to (8);
      \draw[edge] (5) to (6);
      \draw[edge] (6) to (7);
      \draw[edge] (7) to (8);
    \end{tikzpicture}
  \end{figure}
  %%%%%%%%%%%%%%%%%%%%%%%%%%%%%%%%%%%%%%%%%%%%%%%%%%%%%%%%%%%%%%%%%%%%%%%%%%

  \hspace*{0.5cm} \textit{Caso 3.} Este caso lo anexo solo para estar completo,
  pero no debe suceder por la definición de $C(S)$, pues es el polígono convexo formado
  por la envolvente convexa del conjunto de puntos en $S$, así no debe haber un punto
  que no quede interno de $C(S)$ y en consecuencia de $C(S)'$.
  %%%%%%%%%%%%%%%%%%%%%%%%%%%%%%%%%%%%%%%%%%%%%%%%%%%%%%%%%%%%%%%%%%%%%%%%%% FIGURE 2
  \begin{figure}[ht!]
    \centering
    \begin{tikzpicture}
      \node(0) [blueV, label=0:$v_{i-4}$] at (-4, 3){};
      \node(1) [blueV, label=290:$v_{i-3}$] at (-3, 4){};
      \node(2) [blueV, label=270:$v_{i-2}$] at (-2, 4.5){};
      \node(3) [blueV, label=270:$v_{i-1}$] at (-1, 4.8){};
      \node(4) [blueV, label=270:$v_i$] at (0, 5){};
      \node(5) [blueV, label=180:$v_{i+4}$] at (4, 3){};
      \node(6) [blueV, label=250:$v_{i+3}$] at (3, 4){};
      \node(7) [blueV, label=270:$v_{i+2}$] at (2, 4.5){};
      \node(8) [blueV, label=270:$v_{i+1}$] at (1, 4.8){};
      \node(ext) [redV, label=90:$v_{j}$] at (0.5, 6){};
      \draw[edgeC] (0) to (5);
      \draw[edge] (0) to (1);
      \draw[edge] (1) to (2);
      \draw[edge] (2) to (3);
      \draw[edge] (3) to (4);
      \draw[edge] (4) to (8);
      \draw[edgeDotted] (4) to (ext);
      \draw[edgeDotted] (ext) to (8);
      \draw[edge] (5) to (6);
      \draw[edge] (6) to (7);
      \draw[edge] (7) to (8);
    \end{tikzpicture}
  \end{figure}
  %%%%%%%%%%%%%%%%%%%%%%%%%%%%%%%%%%%%%%%%%%%%%%%%%%%%%%%%%%%%%%%%%%%%%%%%%%

  Por lo anterior, hemos llegado a una contradicción en cada caso exhibido. Pues
  ninguno cumple ser un polígono resultado de la envolvente convexa de menor perímetro
  que $C(S)$.

  \[\therefore \text{ C(S) es de menor perímetro entre los que contienen a $S$.}\]
\item Con $C(S)$ el cierre convexo. $C(S)$ es el polígono convexo con área más pequeña, tal que
  $C(S)$ contiene a $S$.
  \newline
  Prueba por \textit{reducción al absurdo}. Analicemos los casos 1 y 2, anteriores y supongamos que
  existe $A(C(S)') < A(C(S))$.
  \begin{itemize}
  \item \textit{Caso 1}. Perdemos convexidad y por tanto llegamos a una contradicción, pues cualquier
    punto interno nos trae como consecuencia la pérdida de convexidad.
  \item \taxtit{Caso 2}. El tener un punto en un segmento no disminuye el área del polígono, por tanto
    contradice que $A(C(S)') < A(C(S))$.
  \end{itemize}
  Por lo anterior, hemos llegado a una contradicción en cada caso exhibido. Pues
  ninguno cumple ser un polígono resultado de la envolvente convexa de menor área
  que $C(S)$.

  \[\therefore \text{ C(S) es de menor área entre los que contienen a $S$.}\]
\end{itemize}
