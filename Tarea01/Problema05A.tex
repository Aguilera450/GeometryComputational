\textbf{Dem. ($a$):} 
Procedemos por inducción.

Observemos que cuando tenemos un solo círculo se cumple por vacuidad. Con $2$ círculos,
  %%%%%%%%%%%%%%%%%%%%%%%%%%%%%%%%%%%%%%%%%%%%%%%%%%%%%%%%%%%%%%%%%%%%%%%%%% FIGURE 1
  \begin{figure}[ht!]
    \centering
    \begin{tikzpicture}
        %\draw [line width=1pt, dash pattern=on 1pt off 2pt] (9,6) circle(1);
        \node(0) at (2, 2){$C_1$}; %[blueV, label=180:$t_1$]
        \node(1) at (6, 2){$C_2$}; %[blueV, label=180:$t_1$]
        \node(t1) [blueV, label=90:$t_1$] at (2, 3){};
        \node(t3) [blueV, label=90:$t_3$] at (6, 3){};
        \node(t2) [blueV, label=270:$t_2$] at (2, 1){};
        \node(t4) [blueV, label=270:$t_4$] at (6, 1){};
        \draw (0) circle (1cm);
        \draw (1) circle (1cm);
        \draw[edge] (t1) to (t3);
        \draw[edge] (t2) to (t4);
     \end{tikzpicture}
  \end{figure}
  %%%%%%%%%%%%%%%%%%%%%%%%%%%%%%%%%%%%%%%%%%%%%%%%%%%%%%%%%%%%%%%%%%%%%%%%%%

$C_1$ y $C_2$, obtenemos
tangentes a cada circunferencia, por cada circuferencia en $2$ puntos distintos. A partir de los puntos de
tangencia con $C_1$ y hasta la frontera externa tenemos un ''pedacito'' de circulo (caso análogo con $C_2$ y
$t_2$, $t_4$). Las tangentes $t_1t_3$ y $t_2t_4$ son las rectas y por tanto terminamos. \newline

\textbf{H.I.} Supongamos que el cierre convexo esta formado por ``pedacitos'' de círculo y rectas, para $k$ círculos. \newline

¿Qué pasa si introducimos un círculo más? Los posibles casos son:
\begin{enumerate}
    %%%%%%%%%%%     1
    \item El nuevo círculo queda contenido en el cierre convexo. En este caso y por H.I tenemos que
    el cierre convexo \textbf{No} se ve modificado y terminamos.
    %%%%%%%%%%%     2
    \item El nuevo círculo queda como frontera. Por H.I, el cierre ya se forma por ''pedacitos''
    de círculos y segmentos de rectas, además el nuevo círculo anexa
    \begin{itemize}
        \item una parte de su contorno y dos segmentos tangentes del anterior cierre al nuevo círculo y por tanto terminamos.
        \item O es parte de la frontera en un solo punto y por \textbf{H.I.} terminamos. \hfill$\square$
    \end{itemize}
\end{enumerate}
