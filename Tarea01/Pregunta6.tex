\large\section*{Pregunta 6}\Large

Prueba o da un contra ejemplo:
\begin{enumerate}[a)]
%%%%%%%%%%%%%%%%%%%%%%%%        Inciso A        %%%%%%%%%%%%%%%%%%%%%%%%%%%
      \item Los vértices convexos de un polígono simple pertenecen al cierre convexo\\
      $\alpha, \beta, \theta$ son vértices convexos, pues $\alpha, \beta, \theta < 180^\circ$ y \textbf{No} forman parte del cierre convexo 
%%%%%%%%%%%%%%%%%%%%%%%%        Inciso B        %%%%%%%%%%%%%%%%%%%%%%%%%%%
      \item La intersección de dos polígonos convexos es convexo.\\
      Procedemos por contradicción.

      Supongamos que la intersección de polígonos \textbf{No} es convexo.
      Entonces $A_i\cap A_j$ con $A_i,A_j$ polígonos convexos tienen al menos
      un ángulo $\beta >180^\circ$, $\beta$ debe ser generado por un punto de
      intersección $A_i\cap A_j$ pues si no, aplicaría que $\beta$ viene de
      un vértice en $A_i$ o $A_j$, pero estos son convexos. Entonces
      ¿Cómo son las intersecciones $A_i\cap A_j$? (Además son solo $2$, por que
      son convexas).\\
      %%%%%%%       Dibujos     %%%%%%%
      Implica que $P_i$ tal que $\measuredangle P_i > 180^\circ$ tiene como
      vecinos a vértices de distintos polígonos digamos $P_{Ai} \in A_i$ y 
      $P_{Aj} \in A_j$ entonces el vecino de $P_i$ que \textbf{No} es $P_{Ai}$
      y esta sobre $A_i$ es un extremo de segmento resultado de prolongar $P_{Ai} P_i$
      (pues ya vimos que el ángulo $P_i$ era forrado por intersección) que tiene
      exactamente $180^\circ$, esto indica que $P_i P_{Aj}$ No pertenece a la 
      intersección$!$.\\

      He aquí una contradicción de suponer que la intersección de polígonos
      convexos \textbf{No} es convexa. Luego concluimos lo que queríamos mostrar.$_\square$
%%%%%%%%%%%%%%%%%%%%%%%%        Inciso C        %%%%%%%%%%%%%%%%%%%%%%%%%%%
      \item La unión de dos polígonos convexos es convexo.\\
      
%%%%%%%%%%%%%%%%%%%%%%%%        Inciso D        %%%%%%%%%%%%%%%%%%%%%%%%%%%
      \item La intersección de un polígono estrellado y un convexo es convexo
\end{enumerate}