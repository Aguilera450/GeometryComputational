\textbf{Dem. ($c$):} Análicemos dos posibles casos:
\begin{itemize}
\item[$\Rightarrow )$] Procedemos por reducción al absurdo.
  
 Nota: $C(S)$ el cierre de $S$ y $C(S')$ el cierre de $S'$
 Supongamos que en $C(S)$ se encuentra $C_i$ como parte de su contorno y $O_i$ 
 (centro de $C_i$). No esta $C(S')$, esto implica que:

 \begin{enumerate}
     \item $O_i$ queda fuera de $C(S')$, esto \textbf{No} se admite, pues $C(S')$
     es el cierre convexo.
     \item $O_i$ \textbf{No} es parte del contorno de $C(S')$ y queda interno, por tanto
     existe $O_j$ que es más ''externo'' que $O_i$, esto implica que $C_j$ es más externo
     que $C_i$ (recordando que los círculos son unitarios) pero $C_i$, es parte del contorno.
     
     He aquí una contradicción de suponer que $O_i\notin C(S')$.
 \end{enumerate}

\item[$\Leftarrow )$] Procedamos por reducción al absurdo.
  
Sea $C(S')$ el cierre convexo de $S'$ y $C(S')$ el cierre convexo de $S$. 
Si $O_i\in C(S')\Rightarrow C_i\notin C(S)$ para algún $i\in [1,...,n]$ y 
$O_i$ corresponde al centro de $C_i$.\\

Como $C_i$ \textbf{No} es parte de $C(S)$, entonces existe $C_j$ más ''externo'' que 
$C_i$ y por tanto $O_j$ es más ''externo'' que $O_i$, pues son círculos unitarios, 
esto contradice el hecho de que $O_i\in C(S')$ refiriendonos a que forma parte de su 
contorno, pues $O_j$ queda fuera de $C(S')$ y $O_j$ es elemento de $S'$. Lo anterior contradice
el hecho de que $O_i$ era parte de la frontera de $C(S').$\\
\end{itemize}
De aquí concluimos lo que se quería mostrar. \hfill $\square$
\newline

