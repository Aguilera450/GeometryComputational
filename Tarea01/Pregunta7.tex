\large\section*{Pregunta 7}\Large

Dado un conjunto de triángulos (definiendo sus 3 puntos) describe un algoritmo 
de barrido de línea que encuentre los triángulos contenidos en otros.\\

Realicemos barrido de línea horizontal.

\begin{enumerate}
    %%%%%%%%%%%%%       1       %%%%%%%%%%%
    \item Ordenemos nuestro conjunto de puntos $O(n\log n)$.
    %%%%%%%%%%%%%       2      %%%%%%%%%%%%
    \item Empezamos el barrido desde el punto más a la izquierda toma $O(n)$
    encontrar el punto más a la izquierda.
    %%%%%%%%%%%%%       3       %%%%%%%%%%%%
    \item Iniciamos barriendo tal como lo realizamos en el barrido de línea para
    encontrar intersecciones, con la modificación, que \textbf{No} sacamos segmentos
    de nuestro árbol $AVL$, en cambio sacamos triángulos.
    %%%%%%%%%%%%%       4       %%%%%%%%%%%%
    \item Por cada vértice en donde inicie un triángulo verificamos el estado de la 
    línea, si hay para ver si ese punto esta contenido en otro triángulo 
    (el que este en el estado), esto es fácil de verificar, pues los triángulos ya
    están definidos. Si el vértice esta contenido en uno o más triángulos, verificamos
    el resto de vértices.
\end{enumerate}

Si No esta contenido en algún triángulo, el triángulo se anexa al estado de la línea
y nos vemos al siguiente vértice.\\

\textbf{Observación} En los $2$ casos metemos el triángulo al estado de la línea.\\

Esto nos puede llevar $O(\frac{n}{3})\in O(n)$. Como esto se realiza para al menos
$\frac{n}{3}$ vértices (los que inician los triángulos), entonces tenemos una complejidad
total de $O(n^2)$.\\

Análisis de complejidad. La complejidad esta contenida en 
$O(n \log n) + O(n) + O(n^2) = O(n^2)$. 