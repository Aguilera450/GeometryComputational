\textbf{2.} Se define el diámetro de un conjunto de puntos $S$ como la distancia más grande
entre cualesquiera dos puntos de $S$, denotado por $d$. Demuestra que d está formado por dos
vértices del cierre convexo de $S$. \newline

\textbf{\textit{Dem.}} Procedamos por inducción en el tamaño de $S$. \newline

Veamos que pasa cuándo $|S| = 3$.
\begin{figure}[ht!]
  \centering
  \begin{tikzpicture}
    %%%%%  Conjunto de puntos  %%%%%
    \node (0) [brownV]  at (1,1){};
    \node (1) [brownV]  at (0,-1){};
    \node (2) [brownV]  at (2,-1){};
    \node (T1)  at (1,-2) {Conjunto S.};

    %%%%% Envolvente convexa  %%%%%
    \begin{scope}[xshift=4cm]
      \node (3) [brownV]  at (1,1){};
      \node (4) [brownV]  at (0,-1){};
      \node (5) [brownV]  at (2,-1){};

      \draw [edge] (3) to (5);
      \draw [edge] (3) to (4);
      \draw [edge] (4) to (5);
      \node (T2)  at (1,-2) {Envolvente convexa.};
    \end{scope}

    %%%%% Componente debajo  %%%%%%
    \begin{scope}[xshift=8cm]
      \node (6) [brownV]  at (1,1){};
      \node (7) [brownV]  at (0,-1){};
      \node (8) [brownV]  at (2,-1){};

      \draw [blueE]  (6) to (7);
      \draw [blueE]  (6) to (8);
      \draw [edge]   (7) to (8);
      \node (T3)  at (1,-2) {Diámetro de S.};
    \end{scope}
  \end{tikzpicture}
\end{figure}
Como podemos notar, el diámetro contiene $3 > 2$ puntos.
\newline

Supongamos que cuando $|S| = k$, el diámetro de $S$ contiene dos puntos del polígono formado
por el cierre convexo de $S$.
\newline

¿Qué pasa cuándo nuestro conjunto de puntos $S$ aumenta en exactamente un punto? Existen 2 casos
interesantes para dar respuesta a esta pregunta, analicemos estos por separado
\begin{itemize}
\item Si el punto esta contenido\footnote{Casos: el punto esta al interior de $C(S)$, el punto esta en un segmento de $C(S)$.} en $C(S)$,
  terminamos. Pues, por hipótesis, nuestro diámetro ya contiene dos puntos en $C(S)$ y la anexión del punto distinguido [digamos $v$] modificaría
  al diámetro solo para aumentar $v$ (* esto lo podríamos  realizar tomando los puntos $x$ y $y$ más cercanos a $v$ que sean parte del diámetro, borar
  $xy$ de diámetro y añadir $xv$ y $vy$ a este).
  
\item Si el punto esta al exterior de $C(S)$, debemos calcular la envolvente convexa de $C(S')$, donde $S' = S + v$, esto lo podemos realizar
  econtrando las tangentes a $v$ con respecto de $C(S)$ y unirlas, este sería el nuevo cierre convexo. Lo anterior implica que $v$ será parte
  del polígono formado por la envolvente convexa (de otro modo, $v$ sería punto interno y se cubre en el caso anterior). Por ($*$) podemos anexar
  $v$ en $d$, y si el punto más cercano a $v$ es un extremo entonces basta con anexar ese segmento a $d$. Supongamos que cuando realizamos la
  unión de $v$ con $C(S)$ perdimos los puntos en $C(S)$ que también estaban en $d$ garantizados por la hipótesis, entonces ya tenemos a $v$ y como
  nadie en la envolvente pertenece a $d$, podemos anexar al vecino de $v$ en $C(S')$ a $d$ y terminamos.
\end{itemize}
