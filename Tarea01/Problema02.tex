\textbf{2.} Se define el diámetro de un conjunto de puntos $S$ como la distancia más grande
entre cualesquiera dos puntos de $S$, denotado por $d$. Demuestra que d está formado por dos
vértices del cierre convexo de $S$. \newline

\textbf{\textit{Dem.}} Procedamos por reducción al absurdo. \newline

Supongamos que el diámetro $C(S)$ no contiene dos puntos del diámetro, entonces, para $d = xy$;
\begin{itemize}
        \item Los dos puntos de $d$ están dentro de $C(S)$.
              
              Esto implica que $d$ no es la distancia más grande entre cualesquiera dos puntos
              de $S$, pues tomando $xv$ donde $v$ es parte de la frontera de $C(S)$ tenemos que
              $||xv|| > ||xy||$. He aquí una contradicción de suponer que $x, y$ están dentro de
              $C(S)$.
        \item Los dos puntos de $d$ están fuera de $C(S)$.
              
              Esto implicaría que $C(S)$ no es el cierre convexo\footnote{Pues $y$ o $x$ queda fuere de $C(S)$ y
              contradice la definición de $C$.} de $S$ y por tanto llegamos a una contradicción.
        \item O hay un punto de $d$ dentro ($x$) y otro fuera ($y$) de $C(S)$.
              
              Esto implicaría que $C(S)$ no es el cierre convexo de $S$ y por tanto hemos llegado a una contradicción.
\end{itemize}
Los casos anterior muestran que $d$ esta formado por dos vértices del cierre convexo. \hfill $\square$
