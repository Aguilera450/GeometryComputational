\textbf{5.} Sea $S$ un conjunto de n círculos unitarios en el plano que posiblemente se intersectan.
Se desea calcular el cierre convexo de $S$.
\begin{itemize}
\item[$a$)] Demuestra que el contorno del cierre convexo de $S$ consiste de segmentos de recta y pedacitos
de círculos en $S$.
\item[$b$)] Demuestra que cada círculo puede estar a lo más una vez en el contorno del cierre convexo.
\item[$c$)] Sea $S$' el conjunto de puntos formado por los centros de los círculos en $S$. Muestra que un
círculo en $S$ aparece en el contorno del cierre convexo sí y sólo sí el centro del círculo pertenece al
cierre convexo de $S$'.
\item[$d$)] Diseña un algoritmo de tiempo $\mathcal{O}(n \log n)$ para calcular el cierre convexo de $S$.
\end{itemize}
