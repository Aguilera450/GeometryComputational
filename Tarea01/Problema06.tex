\textbf{5.} Prueba o da un contra ejemplo:
\begin{enumerate}
\item Los vértices convexos de un polígono simple pertenecen al cierre convexo.
\item La intersección de dos polígonos convexos es convexo.
\item La unión de dos polígonos convexos es convexo.
\item La intersección de un polígono estrellado y un convexo es convexo.
\end{enumerate}
\textbf{Solución:}
\begin{enumerate}
%%%%%%%%%%%%%%%%%%%%%%%%        Inciso A        %%%%%%%%%%%%%%%%%%%%%%%%%%%
\item Los vértices convexos de un polígono simple pertenecen al cierre convexo\\
  %%%%%%%%%%%%%%%%%%%%%%%%%%%%%%%%%%%%%%%%%%%%%%%%%%%%%%%%%%%%%%%%%%%%%%%%%% FIGURE 1
  \begin{center}
    \begin{tikzpicture}[
        my angle/.style = {draw, fill=teal!30,
          angle radius=7mm, 
          angle eccentricity=1.1, 
          right, inner sep=1pt,
          font=\footnotesize} 
      ]
      \draw   (0,2) coordinate[label=below:$A$] (a) --
      (12,2) coordinate[label=below:$C$] (c) --
      (12,7) coordinate[label=above:$B$] (b) --
      (10,3) coordinate[label=above:$D$] (d) --
      (8,5) coordinate[label=above:$E$] (e) --
      (6,3) coordinate[label=above:$F$] (f) --
      (4,5) coordinate[label=above:$G$] (g) --
      (2,3) coordinate[label=above:$H$] (h) --
      (0,7) coordinate[label=above:$I$] (i) -- cycle;
      \pic[my angle, "$\alpha=\SI{45}{\degree}$"] {angle = h--g--f};
      \pic[my angle, "$\beta=\SI{45}{\degree}$"] {angle = f--e--d};
      \draw[edgeDotted] (0,7) to (12,7);
    \end{tikzpicture}
  \end{center}
  %%%%%%%%%%%%%%%%%%%%%%%%%%%%%%%%%%%%%%%%%%%%%%%%%%%%%%%%%%%%%%%%%%%%%%%%%%
  $\alpha, \beta$ son vértices convexos, pues $\alpha, \beta < 180^\circ$ y \textbf{NO} forman parte del cierre convexo
  
  %%%%%%%%%%%%%%%%%%%%%%%%        Inciso B        %%%%%%%%%%%%%%%%%%%%%%%%%%%
\item  Procedemos por contradicción.

  Supongamos que la intersección de dos polígonos $A_i$ y $A_j$ convexos \textbf{No} es convexo.
  Entonces $A_i\cap A_j$ tienen al menos un ángulo $\beta >180^\circ$, $\beta$ debe ser generado
  por un punto de intersección de $A_i\cap A_j$, pues si no, aplicaría que $\beta$ viene de
  un vértice en $A_i$ o $A_j$, pero estos son convexos.

  Entonces ¿Cómo son las intersecciones $A_i\cap A_j$? (Además son solo $2$, por que
  son convexas).\\
  %%%%%%%       Dibujos     %%%%%%%
  Implica que $P_i$ tal que $\measuredangle P_i > 180^\circ$ tiene como
  vecinos a vértices de distintos polígonos digamos $P_{Ai} \in A_i$ y 
  $P_{Aj} \in A_j$ entonces el vecino de $P_i$ que \textbf{No} es $P_{Ai}$
  y esta sobre $A_i$ es un extremo de segmento resultado de prolongar $P_{Ai} P_i$
  (pues ya vimos que el ángulo $P_i$ era forrado por intersección) que tiene
  exactamente $180^\circ$, esto indica que $P_i P_{Aj}$ No pertenece a la 
  intersección$!$.\\

  He aquí una contradicción de suponer que la intersección de polígonos
  convexos \textbf{No} es convexa. Luego concluimos lo que queríamos mostrar.$_\square$
  %%%%%%%%%%%%%%%%%%%%%%%%        Inciso C        %%%%%%%%%%%%%%%%%%%%%%%%%%%
\item La unión de dos polígonos convexos es convexo.
  
  Como podemos ver, las siguientes figuras obedecen a un rectángulo y un triángulo, ambos convexos.
  Sin embargo, su unión no es convexa, pues la línea punteada corta esta unión en $4$ puntos.
  Recordemos que un polígono convexo es cortado a lo más en $2$ puntos, es por esto que el
  enunciado es falso.
  %%%%%%%%%%%%%%%%%%%%%%%%%%%%%%%%%%%%%%%%%%%%%%%%%%%%%%%%%%%%%%%%%%%%%%%%%% FIGURE 1
  \begin{center}
    \begin{tikzpicture}[
        my angle/.style = {draw, fill=teal!30,
          angle radius=7mm, 
          angle eccentricity=1.1, 
          right, inner sep=1pt,
          font=\footnotesize} 
      ]
      % Rectángulo:
      \draw   (0,2) coordinate[label=below:] (a) --
      (0,4) coordinate[label=below:] (c) --
      (4,4) coordinate[label=above:] (d) --
      (4,2) coordinate[label=above:] (b) --
      %(2,3) coordinate[label=above:$H$] (h) --
      %(0,7) coordinate[label=above:$I$] (i) --
      cycle;
      % Triángulo:
      \draw (2,2) coordinate[label=above:] (e) --
      (3.5,3) coordinate[label=above:] (f) --
      (4,6) coordinate[label=above:] (g) -- cycle;
      %\pic[my angle, "$\alpha=\SI{45}{\degree}$"] {angle = h--g--f};
      %\pic[my angle, "$\beta=\SI{45}{\degree}$"] {angle = f--e--d};
      % Recta cortante:
      \draw[edgeDotted] (-2,3) to (5,5);
      % Intersecciones:
      \node(1) [blueV, label=30:$$] at (0, 3.58){};
      \node(2) [blueV, label=30:$$] at (1.45, 4){};
      \node(3) [blueV, label=30:$$] at (3.2, 4.5){};
      \node(4) [blueV, label=30:$$] at (3.8, 4.7){};
    \end{tikzpicture}
  \end{center}
  %%%%%%%%%%%%%%%%%%%%%%%%%%%%%%%%%%%%%%%%%%%%%%%%%%%%%%%%%%%%%%%%%%%%%%%%%%
  %%%%%%%%%%%%%%%%%%%%%%%%        Inciso D        %%%%%%%%%%%%%%%%%%%%%%%%%%%
\item La intersección de un polígono estrellado y un convexo es convexo.
  %%%%%%%%%%%%%%%%%%%%%%%%%%%%%%%%%%%%%%%%%%%%%%%%%%%%%%%%%%%%%%%%%%%%%%%%%% FIGURE 1
  \begin{center}
    \begin{tikzpicture}
      \Star[fill=white!30,draw]{2}{4}
      \draw   (-4,0) coordinate[label=below:] (a) --
      (-4,4) coordinate[label=below:] (c) --
      (4,4) coordinate[label=above:] (d) --
      (4,2) coordinate[label=above:] (b) --
      cycle;
      \draw[edgeDotted] (-3.6,1.5) to (2.5,3);
    \end{tikzpicture}
  \end{center}
  Como podemos notar, la recta punteada corta al polígono resultado de la intersección en
  4 puntos, por esta razón el polígno resultado de la intersección antes mencionada \textbf{No}
  es convexo.
\end{enumerate}

  
