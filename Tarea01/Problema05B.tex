\textbf{Dem. ($b$):} Procedamos por reducción al absurdo.

Supongamos que un círculo puede estar al menos $2$ veces en el contorno del cierre convexo.
Entonces, existe un círculo $C$ que tiene al menos \textbf{2 arcos} sobre el contorno del
cierre convexo. Si esto fuese cierto, tendríamos que hay más de $2$ tangentes sobre $C$ tal
que \textbf{No} se intersectan, como cada tangente a $C$ debe ser parte del contorno,
%%%%%%%%    Dibujo    %%%%%%%%%
 entonces puede pasar que:
%%%%%%%%    Dibujo    %%%%%%%%%
\begin{enumerate}
    %%%%%%%%%%%%    1   %%%%%%%
\item Hay un círculo que \textbf{No} pertenece al cierre, pues es dejado fuera por una tangente (digamos
  que las dos más juntas hacen el cierre). Hasta aquí hemos llegado a una contradicción por suponer que
  hay un círculo con al menos dos arcos en el cierre convexo.
    %%%%%%%%%%%%    2   %%%%%%%%
\item Los círculos son parte del cierre, y por tanto perdemos convexidad. Nuevamente hemos llegado
  a contradecir lo supuesto.
\end{enumerate}

¿Por qué perdemos convexidad?\\
%%%%%%%%    Dibujo    %%%%%%%%%
Como hemos dicho, nuestro círculo esta al menos $2$ veces en el contorno, esto implica que
hay al menos $3$ tangentes al círculo que son parte del contorno y que \textbf{No} son
contiguas (si lo fuera serían una sola recta en vez de $2$). Como $2$ segmentos \textbf{No}
son contiguos, entonces, NO existe el caso donde hayan dos rectas contiguas tales que formen
un ángulo de $180°$
%%%%%%%%    Dibujo      %%%%%%%%
y por tanto una recta (o segmento prolongado) \textbf{siempre} puede cortar los $3$ segmentos
tangentes a $C!$, pues si esto NO sucediera, implicaría que una de las (segmentos) rectas
esta contenida por otras $2$ y esto es falso, pues supusimos que las $3$ son parte del contorno.\\
%%%%%%%%    Dibujo      %%%%%%%%
\[
\therefore \text{ Un círculo esta a lo más una vez en el contorno de cierre convexo.}
\]
