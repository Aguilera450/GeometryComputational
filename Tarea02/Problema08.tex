%%%%%%%%%%%%%%%%%%%% Problema07:
\textbf{8.} Sea S un conjunto de $n$ puntos en el plano y $A$ una subdivisión plana con $\mathcal{O}(n)$
regiones:
\begin{itemize}
\item Muestra que toma $\mathcal{O}(n \log n)$ localizar los puntos de $S$ en $A$.
\item Sea $T$ una triangulación de $S$, ¿cómo podría ayudar esta información para
localizar los puntos de $S$ en $A$? Muestra que a pesar de tener la triangulación
de $S$, la cota sigue siendo logarítmica.
\end{itemize}

\begin{proof}
  Para probar los incisos anteriores basta exhibir un algoritmo que encuentre los puntos
  requeridos en $\mathcal{O}(n \log n)$. A continuación se solucionan los incisos anteriores,
  esto es
  \begin{itemize}
  \item Usando el algoritmo de David Kirkpatrik visto en clase, podemos calcular la localización
    de cada uno de los puntos. Localizar un punto por medio de este algoritmo nos toma $\mathcal{O}(\log n)$,
    cómo buscamos localizar $n$ puntos, entonces tenemos que aplicar este algoritmo $n$ veces. Así,
    localizar la región de $A$ en la que esta contenido cada punto nos cuesta $\mathcal{O}(n \log n)$.
  \item Tener una triangulación en los puntos nos ayudaría a ahorrar algunas subdivisiones, sin embargo,
    habría que verificar que nuestra subdivisión $A$ más $T$ siga siendo plana, por lo que no necesariamente
    ahorramos complejidad. Incluso, si ahorraramos complejidad esta sería pequeña, de hecho sería de tamaño
    constante y nuestro algoritmo seguiría necesitando de $\mathcal{O}(n \log n)$ para localizar todos
    los puntos, pues los demás pasos se siguen teniendo (ir quitando los de vértices de menor grado $< 8$
    e ir retriangulando, luego regresarnos por medio de nuestro árbol).
  \end{itemize}
\end{proof}
