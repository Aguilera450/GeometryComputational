%%%%%%%%%%%%%%%%%%%% Problema02:
\textbf{2.} Muestra que si un polígono tiene $\mathcal{O}(1)$ vértices no regulares,
entonces el algoritmo de triangulación visto en clase se puede adaptar para usar tiempo
$\mathcal{O}(n)$. \newline

$\rhd$ \textbf{Solución:} Recordemos que el algoritmo visto en clase toma
$\mathcal{O}(n + k \cdot \log n)$, dónde $n$ es lo que nos cuesta realizar
el barrido punto a punto (estaciones por eventos) y $k \cdot \log n$ son
los $k$ vértices une y divide (vértices no regulares), y la búsqueda que se
debe hacer en el estado de la línea (cada vértice une y divide) es exactamente
una búsqueda binaria en $\log n$.\newline

Ahora, observemos que $k \in \mathcal{O}(1)$, lo que implicaría que
\begin{eqnarray*}
  k \cdot \log n \approx \log n &\Rightarrow& \mathcal{O}(n + k \cdot \log n) \approx \mathcal{O}(n + \log n)\\
  &\Rightarrow&  \mathcal{O}(n + \log n) = \mathcal{O}(n).
\end{eqnarray*}

\textbf{Obs.} [Acerca del orden] Se considera que tenemos un orden y que nuestro polígono
no es regular. Sin embargo, se me hace curioso saber ¿Cómo obtenemos un orden en tiempo
$\mathcal{O}(n)$? Sabemos que no podemos ordenar por comparaciones en menos que
$\mathcal{O}(n\log n)$, sin embargo, en un barrido de línea no necesitamos ordenar con prioridades
respecto a $(x,y)$ y de hecho basta con tener un orden en $x$ o en $y$. Así, podemos ordenar
basándonos en una proyección, digamos que proyectamos cada punto al eje $X$, esto nos tomará
$O(n)$, luego basta recorrer el eje $X$, en $\mathcal{O}(n)$, y tenemos un orden (No angular),
este orden es suficiente para nuestro barrido de línea. La construcción de la estructura para
el estado de la línea se hará en tiempo $\mathcal{O}(n)$, pues son $n$ puntos y nuestras consultas
son $k \in \mathcal{O}(1)$ sobre el estado de la línea que sabemos tardan $\log n$.\newline

De esta manera, nuestra complejidad queda contenida en $\mathcal{O}(n)$.
\hfill $\lhd$
