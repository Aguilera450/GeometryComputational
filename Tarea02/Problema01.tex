%%%%%%%%%%%%%%%%%%%% Problema01:
\textbf{1.} Sea $P$ un polígono convexo de $n$ vértices.
Suponga que los vértices de $P$ son dados en un arreglo
ordenado. Muestra que, dado unpunto $q$, podemos detectar
si $q \in P$ en tiempo $\mathcal{O}(\log n)$.

\begin{proof}
  Exhibamos un algoritmo en tiempo $\mathcal{O}(\log n)$
  para encontrar que $q \in P$. Esto lo podemos hacer
  de varias maneras, en partícular podemos usar una de
  las estructuras de datos vistas en clase o podemos
  usar una técnica basa en búsqueda binaria. Por simplicidad
  usaré la técnica basada en búsqueda binaria, a continuación
  se muestra el algoritmo deseado:
  \begin{enumerate}
  \item Encontremos el extremo ``más alto'' (digamos $a$) y
    ``más bajo'' (digamos $b$) en $P$ en $\mathcal{O}(1)$,
    pues tenemos los puntos ordenados en un arreglo.
  \item Verificamor que $b.y \leq q.y \leq a.y$ en $\mathcal{O}(1)$.
    Si la condición anterior se cumple, entonces seguimos con
    el algoritmo. Si no se cumple la condición anterior, entonces
    concluimos que $q$ no se encuentra destro de $P$.
  \item Ahora tracemos una recta $\ell$ a partir de $q$ tal que intersecte al
    polígono $P$ respecto a una proyección, digamos la proyección hacia la izquierda.
    Ahora, tendremos tres posibles opciones:
    \begin{enumerate}
    \item $\ell$ intersecta en $0$ puntos (Caso análogo a $2$). Entonces $q \notin P$.
    \item $\ell$ intersecta en $2$ puntos (Caso análogo a $1$). Entonces $q \notin P$.
    \item $\ell$ intersecta en $1$ punto, entonces $q \in P$.
    \end{enumerate}
  \end{enumerate}
    %%%%%%%%%%%%%%%%%%%%%%%%%%%%%%%%%%%%%%%%%%%%%%%%%%%%%%%%%%%%%%%%%%%%%%%%%% FIGURE 1
  \begin{figure}[ht!]
    \centering
    \begin{tikzpicture}[scale=0.70]
      %%%%%%%%%%%% Caso (a)
      \node(0) [blueV, label=150:$v_{1}$] at (0, 0){};
      \node(1) [blueV, label=180:$v_{2}$] at (2, 3){};
      \node(2) [blueV, label=150:$v_{3}$] at (-0.22, 5.83){};
      \node(3) [blueV, label=150:$v_{4}$] at (-3.6, 4.57){};
      \node(4) [blueV, label=150:$v_{5}$] at (-3.46, 0.99){};
      \node(5) [redV, label=150:$q$] at (-4, 2){};
      \draw[edge] (0) to (1);
      \draw[edge] (1) to (2);
      \draw[edge] (2) to (3);
      \draw[edge] (3) to (4); 
      \draw[edge] (4) to (0);
      \draw[edgeDotted] (5) to (-5,2);
      \draw[edge] (-5, -1) to (-5, 6);
      \node (L) at (-3,6){Caso (a)};
      %%%%%%%%%%%% Caso (b)      
      \begin{scope}[xshift=10cm]
        \node(0) [blueV, label=150:$v_{1}$] at (0, 0){};
        \node(1) [blueV, label=180:$v_{2}$] at (2, 3){};
        \node(2) [blueV, label=150:$v_{3}$] at (-0.22, 5.83){};
        \node(3) [blueV, label=150:$v_{4}$] at (-3.6, 4.57){};
        \node(4) [blueV, label=150:$v_{5}$] at (-3.46, 0.99){};
        \node(5) [redV, label=150:$q$] at (2.3, 2){};
        \draw[edge] (0) to (1);
        \draw[edge] (1) to (2);
        \draw[edge] (2) to (3);
        \draw[edge] (3) to (4); 
        \draw[edge] (4) to (0);
        \draw[edgeDotted] (5) to (-5,2);
        \draw[edge] (-5, -1) to (-5, 6);
        \node (L) at (-3,6){Caso (b)};
      \end{scope}
    \end{tikzpicture}
    %%%%%%%%%%%% Caso (c)
    \begin{tikzpicture}[scale=0.70]
      \node(0) [blueV, label=150:$v_{1}$] at (0, 0){};
      \node(1) [blueV, label=180:$v_{2}$] at (2, 3){};
      \node(2) [blueV, label=150:$v_{3}$] at (-0.22, 5.83){};
      \node(3) [blueV, label=150:$v_{4}$] at (-3.6, 4.57){};
      \node(4) [blueV, label=150:$v_{5}$] at (-3.46, 0.99){};
      \node(5) [redV, label=150:$q$] at (0, 2){};
      \draw[edge] (0) to (1);
      \draw[edge] (1) to (2);
      \draw[edge] (2) to (3);
      \draw[edge] (3) to (4); 
      \draw[edge] (4) to (0);
      \draw[edgeDotted] (5) to (-5,2);
      \draw[edge] (-5, -1) to (-5, 6);
      \node (L) at (-3,6){Caso (c)};
    \end{tikzpicture}
  \end{figure}
  %%%%%%%%%%%%%%%%%%%%%%%%%%%%%%%%%%%%%%%%%%%%%%%%%%%%%%%%%%%%%%%%%%%%%%%%%%
  \textbf{Obs.} Encontrar $q$ respecto a los puntos de $P$ se realiza en $\mathcal{O}(\log n)$.
  Producir las recta y encontrar las intersecciones lo realizamos en $\mathcal{O}(1)$. \newline
  
  ¿Por qué será cierto que lo anterior es suficiente? R./ Sabemos que $P$ es convexo,
  entonces toda línea trazada puede cortar a $P$ en a lo más $2$ puntos. Además, como
  la proyección siempre es a la izquierda podemos tomar como referencia los puntos de
  corte respecto al polígono\footnote{Es simple plantear un pequeño sistemas de ecuación que
  me indique en cuántos puntos se corta respecto a $\ell$, todo esto en $\mathcal{O}(1)$.}.
  En otro caso, podemos considerar una orientación distinta para la línea en la que proyectaremos
  $\ell$.\newline
  
  \textit{Análisis de Complejidad.} La complejidad del algoritmo esta contenida en
  \[\mathcal{O}(1) + \mathcal{O}(\log n) \in \mathcal{O}(\log n)\]
\end{proof}
