%%%%%%%%%%%%%%%%%%%% Problema02:
\textbf{3.} Demuestra que cualquier polígono admite una triangulación,
incluso si el polígono tiene hoyos.\newline

\begin{proof}
  Para este ejercicio basta exhibir un algoritmo que genere la triangulación
  del polígno con hoyos. La idea será separar nuestro polígono en polígonos
  sin hoyos y triangularlos por separado. A continuación se exhibe el algoritmo
  deseado:
  \begin{enumerate}
  \item Ordenemos nuestros puntos. Esto lo realizamos en $\mathcal{O}(n \log n)$.
  \item Realizamos un barrido de línea para encontrar los hoyos. Este barrido
    nos tomará $\mathcal{O}(n)$, y detectaremos hoyos cuando el estado de la
    línea nos indique que hay al menos $4$ aristas en su posición y conforme
    el barrido avance esto se sigue cumpliendo hasta que eventualmente llegamos
    a que $2$ de los (al menos) $4$ segmentos que siempre estuvieron en el estado
    de la línea (no necesariamente los mismos segementos), siempre fueron ``internos''
    de los otros segmentos (siempre estuvieron ``entre ellos'').

    Una manera de hacer esto sería coloreando las aristas, si en un punto el estado
    de la línea me indica que las trayectorias formadas hasta el momento (con el
    avance de la línea) y coloreadas de distinto color se unen, entonces cambio el
    color a la trayectoria más pequeña. Si una trayectoria núnca se unió y además
    forma un ciclo interno al polígono, entonces este es un hoyo. \textbf{Obs.} Tomar
    esta idea sin mayor cuidado nos puede llevar a una complejidad cuadrada (aquí
    podemos hacer uso de la estrategia usada en el agoritmo de  Borůvka-Kruskal
    para encontrar ciclos y agrupar).
  \item Ahora que hemos encontrado los ciclos, podemos con gran facilidad segmentar
    nuestro polígono en nuevos polígonos sin hoyos. Para esto;
    \begin{enumerate}
    \item tomamos una arista del hoyo $O_i$, digamos $e$, entonces unimos los extremos
      de $e$ con los vértices más cercanos en el polígono y verificamos si estas nuevas
      aristas son ``estrictamente internas al polígono'' o no. En caso de que sí, entonces
      serán parte de nuestra subdivisión, en otro caso la descartamos. Esto lo podemos
      hacer con ayuda de un nuevo barrido y nos toma $\mathcal{O}(n)$ barrer nuestros puntos
      y $\mathcal{O}(n)$ realizar las comparaciones para encontrar las nuevas aristas para
      nuestra subdivisión. Si el número de aristas de $O_i$ es lineal respecto al número de vértices
      en el polígono, entonces tendremos que hacer $\mathcal{O}(n^2)$ comparaciones. De lo
      anterior tenemos una subdivisión.
    \end{enumerate}
  \item Ahora aplicamos el algoritmo de triangulación visto en clase a cada parte de la subdivisón
    encontrada.
  \end{enumerate}
  \textit{Análisis de complejidad.} La complejidad de este algoritmo esta contenida en
  \[\mathcal{O}(n + k\cdot \log n) + \mathcal{O}(n) + \mathcal{O}(n \log n) + \mathcal{O}(n^2) = \mathcal{O}(n^2).\]
  
  Como el algoritmo anterior nos regresa una triangulación de nuestro polígno con hoyos, entonces
  hemos mostrado lo requerido.
\end{proof}
