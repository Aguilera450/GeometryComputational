%%%%%%%%%%%%%%%%%%%% Problema07:
\textbf{7.} Da un algoritmo que calcule en tiempo $O(n \log n)$ una diagonal que divida un
polígono simple con $n$ vértices en dos polígonos simples cada uno con a lo más
$\lfloor \frac{2n}{3} \rfloor + 2$ vártices. \textbf{Hint}: Usa la gráfica dual de la triangulación.
\newline

$\rhd$ \textbf{Solución:} A continuación se exhibe un algoritmos que resuelve este problema
\begin{enumerate}
\item Obntener la triangulación de nuestro polígono. Esto lo realizamos en $\mathcal{O}(n \log n)$
  con el algoritmo visto en clase.
\item Sabemos que toda triangulación tiene $n - 3$ aristas y por tanto nuestra gráfica dual que
  se formará a partir de cada arista interna será de un orden $\mathcal{O}(n)$. Entonces, la
  construcción de la gráfica dual nos toma $\mathcal{O}(n)$. Además, la gráfica dual es un árbol.
\item Encontramos los puntos más lejanos en una gráfica. Esto, por algoritmos $I$ sabemos que
  lo podemos hacer en $\mathcal{O}(n)$ para un árbol. La técnica se basa en ``colgar'' el árbol
  por medio de un recorrido a profundidad desde la raíz hasta la hoja de mayor profundidad y
  tomar esa hoja como la nueva raíz y volver a recorrer hasta la hoja más profunda.
\item  Recorrer la gráfica dual iniciando desde cada uno de los puntos más lejanos. Eligimos una
  orientación y siempre seguimos esta orientación (La orientación, la podemos tener con DCEL) para
  el reccorido. Así evitaremos dividir el polígono en más de $2$ polígonos nuevos. Esto lo hacemos
  en $\mathcal{O}(n)$, pues el orden de las aristas de nuestra gráfica dual es lineal.
\item Durante cada recorrido colorearemos los vértices del triángulo en el que estemos, todos
  del mismo color. En el momento que hayamos coloreado, exactamente $\lfloor \frac{2n}{3} \rfloor + 2$
  vértices. Entonces regresamos la arista $e_T$ de la triangulación por la que fue creada la arista
  siguiente a la última hasta haber coloreado los  $\lfloor \frac{2n}{3} \rfloor + 2$ vértices.
  Esto lo hacemos en $3 \cdot \left( \lfloor \frac{2n}{3} \rfloor + 2\right)$ pasos que obedece
  a un orden líneal.\newline
  
  ¿Con cuál recorrido nos quedamos? Con el que llegue primero a colorear los $\lfloor \frac{2n}{3} \rfloor + 2$
  vértices y cumpla que la arista divide al polígono en las características deseadas.\newline

  \textbf{Obs.} Podríamos hacer una versión con $\lfloor \frac{n}{3} \rfloor - 2$ tomandolo por cota.
\end{enumerate}
Al final, $e_T$ es la diagonal búscada.\newline

\textit{Análisis de la complejidad.} La complejidad del algoritmo esta contenida en
\[\mathcal{O}(n \log n) + \mathcal{O}(n) = \mathcal{O}(n \log n).\]
\hfill $\lhd$
